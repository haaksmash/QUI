% !TEX TS-program = pdflatex
% !TEX encoding = UTF-8 Unicode

% This is a simple template for a LaTeX document using the "article" class.
% See "book", "report", "letter" for other types of document.

\documentclass{article} % use larger type; default would be 10pt

\usepackage[utf8]{inputenc} % set input encoding (not needed with XeLaTeX)

%%% Examples of Article customizations
% These packages are optional, depending whether you want the features they provide.
% See the LaTeX Companion or other references for full information.

%%% PAGE DIMENSIONS
\usepackage{geometry} % to change the page dimensions
\geometry{a4paper} % or letterpaper (US) or a5paper or....
\geometry{margin=1in} % for example, change the margins to 2 inches all round
% \geometry{landscape} % set up the page for landscape
%   read geometry.pdf for detailed page layout information

\usepackage{graphicx} % support the \includegraphics command and options

% \usepackage[parfill]{parskip} % Activate to begin paragraphs with an empty line rather than an indent

%%% PACKAGES
\usepackage{booktabs} % for much better looking tables
\usepackage{array} % for better arrays (eg matrices) in maths
\usepackage{paralist} % very flexible & customisable lists (eg. enumerate/itemize, etc.)
\usepackage{verbatim} % adds environment for commenting out blocks of text & for better verbatim
\usepackage{subfig} % make it possible to include more than one captioned figure/table in a single float
% These packages are all incorporated in the memoir class to one degree or another...

%%% HEADERS & FOOTERS
\usepackage{fancyhdr} % This should be set AFTER setting up the page geometry
\pagestyle{fancy} % options: empty , plain , fancy
\renewcommand{\headrulewidth}{0pt} % customise the layout...
\lhead{QUI}\chead{}\rhead{J.H., H.S.}
\lfoot{}\cfoot{\thepage}\rfoot{}

%%% SECTION TITLE APPEARANCE
\usepackage{sectsty}
\allsectionsfont{\sffamily\mdseries\upshape} % (See the fntguide.pdf for font help)
% (This matches ConTeXt defaults)

%%% ToC (table of contents) APPEARANCE
\usepackage[nottoc,notlof,notlot]{tocbibind} % Put the bibliography in the ToC
\usepackage[titles,subfigure]{tocloft} % Alter the style of the Table of Contents
\renewcommand{\cftsecfont}{\rmfamily\mdseries\upshape}
\renewcommand{\cftsecpagefont}{\rmfamily\mdseries\upshape} % No bold!

%%% END Article customizations

%%% The "real" document content comes below...

\title{Query Universal Interface}
\author{Haak Saxberg and Jess Hester}
\date{November 5, 2011} % Activate to display a given date or no date (if empty),
         % otherwise the current date is printed 

\begin{document}
\maketitle

\section{Domain}
Different database backends have different interfaces. This means that it takes a lot of work to transition from one database
to another database; in the case of SQL-compliant interfaces, like MySQL and PostgreSQL, this work is manageable, since these
backends have overlapping syntaxes and very simple representation models (rows with columns, stored in tables). However, if one
needed to switch between non-SQL backends, the task is nontrivial. Two of the biggest non-relational databases (Google's AppEngine
and Amazon's SimpleDB) have very different APIs; migrating an application between the two is full of headaches. A DSL that abstracts
the interaction between an application and the database API would alleviate that headache, making migration of code a much smoother
process.
\section{Language Overview}
\subsection{Basic Computation}
Converting instances of classes into database-intelligible formats and vice-versa. In a relational database scheme, this would mean making a 
table for each class, and columns for each attribute of the class (and being able to reverse that translation as necessary). In non-relational
schemes, this could be very similar or very different; Amazon's S3 service uses the concept of ``buckets'' for data, while Google's AppEngine
stores its data in a format much more like traditional tables.
\subsection{Basic Data Structures}
Classes themselves are the data structures we're interested in, since those are the things we want to store in the databases. In order to keep
classes and their database representations separate, we'll probably have ``Mapper'' structures who do the lifting. If we have time,
then we might also have Syntax Trees, which would allow us to parse (limited) ``raw'' queries written in SQL (or some other standard ) into 
actionable code that can talk to the proper backend. 
\subsection{Basic Control Structures}
Users of our language will be able to filter results by chaining filtering methods on the result of some query. If we have time, as we said before,
they'll be able to write ``raw'' queries, which will grant them a greater amount of control over their interaction with the database. (This is a pipe dream)
\subsection{Input and Output}
Since this is an internal DSL, it acts on already extant classes and their instances; by introspecting a class, we can find its attributes.
These can then be stored in the appropriate place, and retrieved as needed - storage and retrieval methods are thus the inputs of our
DSL. Output of storage routines is either nothing if successful or an exception if something went wrong. In the retrieval routines, output is
either an instance of the class if successful or an exception if unsuccessful.
\subsection{Error Handling}
As mentioned above, there could be a number of problems encountered; issues with retrieval could include no database found, no data in the database
to retrieve, no data that matches the definition of the class we're trying to retrieve to, or interrupted connections to the database. With storage, we also
have to deal with non-existent or interrupted database connections, as well as appropriate locations not existing for the data of the instance being stored.
The traditional way to handle these is to throw informative exceptions - then the user can catch them and perform recovery as they see fit.
\subsection{Other DSLs in same domain?}
Yes, but none that focus on universalizing interaction with \emph{non-relational} databases; there are many that attempt to bridge the gaps between
relational databases (ala SQLAlchemy or Django's own ORM), but there doesn't seem to be a large market for non-relational bridgers. At least, not a
publicly available market.
\section{Implementation Plan}
We're leaning strongly towards Python as the host language; there are already low-level APIs for both of our focus databases available in Python, and we're
more comfortable in that language than in Scala or Java. Python's decorator concepts also lend themselves to a ``store this, not that'' kind of model, and 
hopefully mean that we won't have to force users to redeclare their classes - they'd just decorate them with our decorators.\\
There are APIs available for Java as well, but since we haven't seen much of the way that Scala can `pull out' to Java, we're hesitant to move in that direction.
\section{Teamwork Plan}
FUN VERSION:\\
I mean, we're both going to work on it. Neither of us trusts the other one enough to not do that.\\
Since this is a new area for both of us, we're probably going to use a kind of pair programming -- always working in the same room, at the least -- so that we can 
have both pairs of eyes watching for insidious logical errors. All planning and design is going to be done jointly, using the democratic system of voting. In the event
of a tie, a gladiatorial battle will take place. The winner survives and gets to choose the design path. The more serious the design choice, the greater the spectacle 
(core language decisions will be decided by a naval battle hosted in Case's courtyard).

SERIOUS VERSION:\\
Language planning and design will always be done jointly. Pair programming will be employed to ensure equal division of labor. While one person is implementing, the 
other is documenting the process for use in the final documentation. This will ensure that we are both acquainted with all aspects of the language we're
implementing, as well as allowing the documenter to check the progress of the implementer against the design plan as we go.
\end{document}
